\chapter{Combinatorics}


\section{Basic Review}

\begin{summary}
	One possible interpretation for the formula of $ n $-choose-$ k $ is the following. Let $ A = \set{a,b,c,d,e} $, and assume we want to choose $ 2 $ elements from the set. Fix some ordering for the elements in the set and assume each selection is represented by a 5-tuple, where each index specifies if that elements in the set is chosen. For instance $ (1,1,0,0,0) $ corresponds to the choice $ \set{a,b} $. So the total number of such choices will be total number of ways that we can arrange two 1's and three 0's, which is
	\[ \frac{5!}{2! 3!}. \]
	So in general we can write
	\[ \frac{n!}{(n-k)! k!}. \]
\end{summary}




\section{Solved Problems}
\begin{problem}
	What is the number of choosing $ k $ objects out of $ n $, where order does not matter, but repetitions are allows.
\end{problem}
\begin{solution}
	This problem is very similar to the one in the thermal physics bo by Schroeder when studying the number of possible ways to distribute $ Q $ units of energy in $ N $ Einstein solids. 
	
	Let's consider a concrete example where we want to choose 3 objects from $ \set{a,b,c,d,e} $ with replacement and order is not important. Then assume each object in the set is a container, and we have 3 balls to put in containers (exactly the same as distribution energy units between Einstein solids). So the outcome $ aaa $ corresponds to putting all three balls in $ a $, and etc. One can represent each outcome with a dot-line diagram. For instance $\bullet\ \bullet\ \bullet\ |\ |\ |\ $ corresponds $ aaa $ outcome. Note that we have $ n-1 $ lines and $ k $ balls. So total number of ways to arrange these objects is
	\[ \frac{(n+k-1)!}{k!(n-1)!} = \binom{n+k-1}{k}. \]
\end{solution} 
\begin{remark}
	Interestingly, the Einstein solid problem, and number of ways that one can choose $ k $ scoops of ice-cream in a shop with $ n $ scoops of ice-cream is the same.
\end{remark}